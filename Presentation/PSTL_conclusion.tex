\section{Conclusion}

\begin{frame}
	\frametitle{Bilan}
	\begin{itemize}
		\item Étude d'articles scientifiques
		\item Complexité linéaire
		\item Modules réutilisables
	\end{itemize}
\end{frame}

\begin{frame}
	\frametitle{Pour la suite}
	\begin{itemize}
		\item Extension aux graphes
		\item Optimisation mémoire
		\item Prise en charge du format ARB
		\item Ajout de critères de représentation %minimisation de largeur
	\end{itemize}
\end{frame}

%\chapter{Conclusion}
%
%\section{Bilan}
%\paragraph{}L'application \verb|treeDisplay| permet de calculer les coordonnées des n\oe uds d'un arbre en temps linéaire par rapport à ce nombre de n\oe uds. Cependant, l'affichage qui en résulte n'est pas aussi élégant que GraphViz. Nous ne pouvons par exemple pas voir la structure de l'arbre en détail. Mais puisque nous travaillons majoritairement avec des arbres de grande taille, les détails nous intéressent peu. On veut avant tout observer des tendances. 
%\paragraph{} Nous avons également observé que la partie de l'application qui prend le plus de temps à l'exécution est le parsing et la génération de sortie. La clé de la complexité de cette application était donc de lire les fichiers d'entrée et de générer les fichiers de sortie de manière optimale, de façon à utiliser le moins de ressources possible.
%
%\section{Pour la suite}
%
%\paragraph{}La structure choisie permet aussi de représenter des graphes. On peut donc envisager par la suite d'implémenter un algorithme de calcul de coordonnées pour les graphes. Les modules de parsing et de génération sont pleinement réutilisables.
%
%\paragraph{}La complexité en mémoire est actuellement égale à celle en temps. Cependant, on utilisant une table de hachage pour les sous-arbres, on peut devenir sous-linéaire. L'idée consiste à calculer des coordonnées relatives et lorsqu'un arbre comporte deux (ou plus) sous-arbres identiques. Le sous-arbre en question n'est sauvegarder en mémoire qu'une seule fois et la deuxième fois on ne sauvegarde qu'une référence. On calcule ensuite les coordonnées absolues lors de la génération.